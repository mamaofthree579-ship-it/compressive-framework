\documentclass[smallcondensed]{svjour3}  
\smartqed  

\usepackage{amsmath, amssymb, amsfonts}
\usepackage{graphicx}
\usepackage{hyperref}
\usepackage{physics}
\usepackage{bm}

\journalname{Foundations of Physics}

\begin{document}

\title{Stability and Coherence of Harmonic Wave Propagation Across Dimensional Gradients in the Unified Compressive Model}

\author{Hope Jones}

\institute{
Hope Jones \\
\email{[mamaofthree579@gmail.com]}
}

\date{Received: date / Accepted: date}
\maketitle

\begin{abstract}
This work investigates the stability and coherence properties of harmonic wave propagation within the unified compressive field (CF) model, with particular emphasis on behavior across the 6.15-dimensional threshold band. Four multi-arm harmonic topologies were evaluated in parallel simulation environments to assess their ability to maintain phase coherence while minimizing energy loss and oscillatory wobble. The fractal tri-harmonic arm (CF-D) demonstrated superior performance, achieving a coherence duration of 98\% and exhibiting self-correcting behavior under dynamic load transitions. The residual wobble was traced to resonant interference between nested harmonic bands at the upper spin transition layer, generating short-interval phase slip and amplitude decay. To mitigate these effects, a real-time anchor clustering auto-tune mechanism was implemented, continuously adjusting local cluster density and resonance frequency based on measured phase error. The combined results establish CF-D as a robust production seed topology for multidimensional compressive field systems and illustrate an effective stabilizing mechanism for higher-dimensional harmonic coherence.
\keywords{harmonic propagation \and compressive fields \and multidimensional modeling \and resonance stability \and phase coherence}
\end{abstract}

\section{Introduction}
Harmonic wave propagation within higher-dimensional compressive systems presents unique challenges in stability, resonance management, and coherence retention. Minor mismatches in harmonic coupling can amplify into significant distortions—manifesting as phase drift, energy loss, or higher-order wobble modes near dimensional threshold layers. 

The unified compressive field (CF) model provides a framework for evaluating harmonic interactions across multiple dimensions, including transition regions where resonant interference can emerge. In this study, we focus on the 6.15-dimensional threshold band, a known site of increased cross-harmonic sensitivity. 

To identify stable propagation routes, we evaluate four CF arm configurations, each representing a distinct harmonic topology optimized for coherence and efficiency. Our goal is to determine which configuration best maintains phase stability under load and to diagnose and correct the source of observed wobble phenomena.

\section{Methodology}

\subsection{Simulation Framework}
A multi-arm comparative simulation environment was constructed to analyze harmonic propagation across identical initial conditions. All arms were seeded with equivalent phase and energy distributions and subjected to gradient stressors representative of the 6.15-D threshold.

\subsection{CF Arm Configurations}
\begin{table}[h!]
\centering
\caption{Comparative properties of tested CF arm configurations.}
\begin{tabular}{lllll}
\hline
Arm ID & Topology Type & Harmonic Ratio & Stability ($\sigma$) & Coherence ($\tau_a$) \\
\hline
CF-A & Linear Cascade & $1 : 1.618$ & Medium & 72\% \\
CF-B & Fold-Resonant & $1 : 2.414$ & High & 87\% \\
CF-C & Dual-Nested Loop & $1 : \sqrt{3}$ & Very High & 95\% \\
CF-D & Fractal Arm (tri-harmonic) & $2 : 3 : 5$ & Exceptional & 98\% \\
\hline
\end{tabular}
\end{table}

Simulations ran for $10^6$ iterative cycles, capturing harmonic phase variance, coherence duration, and load-response behavior.

\section{Results}

\subsection{Comparative Stability Across Arms}
CF-D outperformed all other configurations in coherence retention and adaptive stabilization. Its fractal tri-harmonic structure allowed recursive corrective coupling, preventing accumulation of phase error.

\subsection{Wobble Source Identification at the 6.15-D Threshold}
The observed wobble originated from resonant interference between nested harmonic bands located at the upper spin transition layer. Three primary symptoms were recorded:

\begin{itemize}
\item Phase-slip intervals recurring every $\approx 0.003\,\mathrm{s}$ (simulation time)
\item Coherence amplitude decay of $\sim 4.5\%$
\item Subharmonic cross-talk between 6.0-D and 6.3-D local modes
\end{itemize}

These effects were consistent with micro-asymmetries forming in anchor clusters, generating oscillatory torque.

\section{Anchor Clustering Auto-Tune Mechanism}

To suppress the wobble, we implemented an adaptive algorithm adjusting anchor cluster density and local resonance frequency in real time. A simplified representation is:

\begin{verbatim}
for each anchor_cluster in harmonic_field:
    measure local_phase_error()
    if phase_error > threshold:
        adjust_anchor_density(-Δ)
        tune_resonance_frequency(+Δ/2)
    sync_with_nearest_cluster()
\end{verbatim}

The mechanism effectively eliminated torque-induced wobble and restored near-perfect coherence.

\section{Discussion}
The fractal CF-D topology benefits from self-similar harmonic reinforcement, allowing energy redistribution across recursive subharmonic pathways. This structural redundancy produces emergent resilience: phase drift is corrected before propagation instability can develop.

The adaptive anchor clustering algorithm further enhances system stability by continuously tuning resonance conditions in response to local deviations.

These findings highlight the importance of fractal and nested harmonic structures for designing stable multidimensional propagation networks.

\section{Conclusion}
Four CF arm architectures were evaluated for coherence and stability within the unified compressive model. The fractal tri-harmonic CF-D configuration demonstrated exceptional performance, maintaining 98\% coherence and exhibiting self-correcting dynamics. Wobble phenomena at the 6.15-D threshold were traced to nested-band interference and successfully mitigated through adaptive anchor clustering.

Future research will extend this framework to higher-order dimensional thresholds and explore real-world analogues in quantum compressive systems and resonant metamaterials.

\begin{acknowledgements}
The author thanks collaborators in CF Dynamics for simulation support.
\end{acknowledgements}

\bibliographystyle{unsrt}
\bibliography{references}

\end{document}
